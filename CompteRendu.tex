%Classe du document, A4, police taille 12
\documentclass[a4paper,12pt]{article}

% Dictionnaire français, pour caractères spéciaux, tirets, caractères accentués
\usepackage[french]{babel}
\usepackage[utf8]{inputenc}
%Toujours plus d'accents
\usepackage[T1]{fontenc}

\usepackage{indentfirst}
%Pour créer des paragraphes random
\usepackage{lipsum}  
%bibliographie
%Le style dépend du projet, voir avec le grand chef
%A mettre à l'endroit où vous voulez la faire apparaitre
%Donc dans le code pas ici t'as vu
%\bibliographystyle{ieeetr}
%\bibligraphy{nom_du_fichier.bib}

%hauteur entre deux lignes
\baselineskip 200cm
%hauteur entre deux paragraphes
\parskip 2mm
%longueur d'indentation
\parindent 2mm
%(on utilise \indent et \noindent sinon)

%Gérer ses marges

%Facilement
\usepackage[margin=2cm]{geometry}

%Précisément
%\usepackage[left=2cm , right=2cm, bottom=2cm, top=2cm, headheight=2cm]{geometry} 
%header c'est l'en-tête pas la marge supérieure

%Toujours plus précisément
%\addtolength{\oddsidemargin}{-0.5in}
%\addtolength{\evensidemargin}{-5cm}
%\addtolength{\topmargin}{-0.5in}

%Faires des articles  plusieures colonnes
\usepackage{multicol}
%Separation des colonnes
\setlength{\columnsep}{2cm}

%Avoir des entêtes et pieds de page stylés
\usepackage{fancyhdr}
\pagestyle{fancy}
%Pour enlever l'entête avec les sections
%\fancyhf{}

%Ca se fait sous format \<pos><type>{<contenu>}
%type c'est "head" ou "foot"
%pos pour position gauche "l", droite "r" ou centre "c"
%contenu c'est ce que tu mets dans dedans 
%marche aussi avec des images mais flemme
%mettre un trait
%\renewcommand{\footrulewidth}{1.5pt}

% Liens dans le document
\usepackage{hyperref}  
% Légendes dans les environnements "figure" et "float"
\usepackage{subcaption}
%La base pour faire des figures juste
\usepackage{graphicx}
\usepackage[export]{adjustbox}
\usepackage{wrapfig}
%Trucs utiles pour les maths
\usepackage{amsfonts}
\usepackage{amsmath}
\usepackage{ntheorem}
\newtheorem{definition}{Définition}
\newtheorem{proposition}{Proposition}

\usepackage{listings}


\begin{document}

\begin{titlepage}
    \begin{center}
        \vspace*{0.5cm}
        {\Large ÉCOLE NATIONALE DES PONTS ET CHAUSSÉES}\\
        \vspace{1cm}
        \rule\linewidth{0.05cm}
        {\huge Résolution d'une équation aux dérivées partielles par processus gaussien\par}
        \rule\linewidth{0.05cm}
        \vspace{1cm}
        {\Large Projet 1A\par}
        \vspace{0.8cm}
        {\Large Par}\\
        \vspace{0.3cm}
        {\large \textit{Nils Baulier, Raphaël Viard, Salma Kably et Simon Weissberg}}\\
        \vspace{0.8cm}
        {\Large Supervisé par}\\
        \vspace{0.3cm}
        {\large \textit{Urbain Vaes}}\\
        \vspace{0.7cm}
        \includegraphics[scale=0.2]{logo_ponts.jpg}\\
    \end{center}
\end{titlepage}

\tableofcontents

\section{Introduction}

Historiquement, les processus de régression gaussienne furent introduits par Krige en 1960,
 cherchant à l’époque à obtenir la distribution d’or dans un milieu, en ne se basant que 
 sur des échantillons. Aujourd’hui, ils permettent par exemple d’étudier la perméabilité 
 radioactive d’un sol. L'approche de résolution d'équations aux dérivées partielles par 
 processus gaussien est une technique d'approximation de la solution par une fonction ayant
 un caractère aléatoire, permettant par exemple de modéliser l'incertitude sur une donnée du problème. 
 Avant de se lancer dans une telle résolution, il parait naturel d'abord de simplement essayer 
 d'approcher une fonction connue par un processus gaussien (c'est la régression gaussienne), 
 ce qui sera l'objet principal de ce premier rendu. Cette première étape nous orientera pour 
 ensuite résoudre des équations différentielles linéaires.

\section{Processus gaussien}

\subsection{Définition}

\begin{definition}
    Un processus stochastique est une famille de variables aléatoires $({X_t})_{t \in T}$ 
    définies sur le même espace de probabilité $(\Omega,\mathcal{F},\mathcal{P})$ indexée par 
    $T$ et à valeurs dans $S$. Un processus stochastique peut donc aussi être vu comme une application 
    $X:\Omega \times T \to S$ tel que 
    pour tout $t$ dans $T$, $X( . ,t)$ est une variable aléatoire.

    Un processus gaussien est un processus stochastique $({X_t})_{t \in T}$ tel que 
    $\forall n \in \mathbb{N}, \forall (t_1,t_2,...,t_n) \in T^{n}, (X_{t_1},...,X_{t_n})$
     est un vecteur gaussien.
\end{definition}

\begin{proposition}
    Soit $({X_t})_{t \in T}$ un processus gaussien. Alors il existe $\bar{m}:T \to \mathbb{R}$ et
     $\bar{k}:T \times T \to \mathbb{R}$ définie positive tel que $\forall n \in \mathbb{N},
      \forall (t_1,t_2,...,t_n) \in T^{n}, (X_{t_1},...,X_{t_n}) 
      \sim \mathcal{N}( (\bar{m}(t_1) , ... , \bar{m}(t_n))^\top , K_{t_1,t_2,...,t_n} )$ 
      où $K_{t_1,t_2,...,t_n}$ est la matrice de coefficients $K_{i,j}=\bar{k}(t_i,t_j)$.
    
    Réciproquement, s'il existe un tel m et un tel c, alors $({X_t})_{t \in T}$ est un processus gaussien. On note alors $({X_t})_{t \in T} \sim \mathcal{GP}(\bar{m},\bar{k})$ et on appelle $\bar{m}$ la fonction moyenne et $\bar{k}$ le noyau de covariance du processus gaussien.
\end{proposition}

\subsection{Modélisation d'une trajectoire d'un processus gaussien}

\end{document}